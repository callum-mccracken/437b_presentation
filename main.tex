% !TEX program = lualatex
\documentclass[10pt, aspectratio=169]{beamer}

\usepackage{appendixnumberbeamer}
\usepackage[export]{adjustbox}
\usepackage{booktabs}
\usepackage[scale=2]{ccicons}
\usepackage{pgfplots}
\usepackage{xspace}
\usepackage[compat=1.1.0]{tikz-feynman}

\usetheme[progressbar=frametitle,
subsectionpage=progressbar,
block=fill]{metropolis}

% make progress bar for every slide rather than just sections
\makeatletter
\setlength{\metropolis@frametitle@padding}{1.8ex}% <- default 2.2 ex
\setbeamertemplate{frametitle}{%
    \nointerlineskip%
    \begin{beamercolorbox}[%
        wd=\paperwidth,%
        sep=0pt,%
        leftskip=\metropolis@frametitle@padding,%
        rightskip=\metropolis@frametitle@padding,%
        ]{frametitle}%
        \metropolis@frametitlestrut@start%
        \insertframetitle%
        \nolinebreak%
        \metropolis@frametitlestrut@end%
    \end{beamercolorbox}
    \usebeamertemplate*{progress bar in head/foot}
}
\setlength{\metropolis@progressinheadfoot@linewidth}{1pt}



\usepgfplotslibrary{dateplot}

\newcommand{\themename}{\textbf{\textsc{metropolis}}\xspace}

\title{\Huge \textnormal{Improving Background Estimation for Di-Higgs Searches with ATLAS}}
\subtitle{PHYS 437B Presentations\newline
13 January, 2020}
% \date{\today}
\date{}
\author{Callum McCracken\\
Supervisor: Maximilian Swiatlowski\\
Co-Supervisor: Eduardo Martin-Martinez\\
Collaborators: Todd Seiss, Mel Shochet}

%\titlegraphic{\hfill\includegraphics[height=1.5cm]{logo.pdf}}
\setbeamercolor{background canvas}{bg=white}

\begin{document}

\maketitle

%%%%%%%%%%%%%%%%%%%%% SLIDE 1
{\setbeamertemplate{frame footer}{Image sources:
  \href{https://public-archive.web.cern.ch/en/research/AccelComplex-en.html}{\alert{Accelerator Complex}}, 
  \href{https://atlas.cern/discover/detector}{\alert{ATLAS}}, 
  \href{https://en.wikipedia.org/wiki/Higgs_boson}{\alert{Standard Model}}}
\begin{frame}{Overview: Higgs Research}
  \begin{columns}[onlytextwidth]
    \begin{column}{0.4\textwidth}
      \centering
      \includegraphics[width=0.9\linewidth]{images/accelerator_complex.png}
      \includegraphics[width=0.9\linewidth]{images/ATLAS_full.png}
    \end{column}
    \begin{column}{0.6\textwidth}
      \includegraphics[width=0.9\linewidth]{images/Standard_Model_of_Elementary_Particles.png}
    \end{column}
​  \end{columns}
\end{frame}
}


%%%%%%%%%%%%%%%%%%%%% SLIDE 2
\begin{frame}{The Big Picture -- Measuring the Higgs Self-Coupling}
  Relevant section of the SM Lagrangian for Higgs potential:\\
  
  $$V(\phi) = -\mu^2 \phi^2 + \lambda \phi^4 + \ldots \text{ Taylor exp. at min } \to V_T(\phi) = -\frac{\mu^4}{4\lambda} + \frac{\sqrt{2}\mu^3}{\lambda} \phi - 4 \mu^2\phi^2 + 2\sqrt{2\lambda}\mu\phi^3 + \ldots$$
  
  Constant and $\phi$ terms: can eliminate with change of coordinates, $\phi^2$: mass term,\\
  $\phi^3$: self-interaction or \alert{self-coupling} term, not well constrained\\ (current best: $\kappa_\lambda = (2\sqrt{2\lambda}\mu)/(2\sqrt{2\lambda}\mu)_{\text{SM}}$, $\kappa_\lambda \in [-2.3, 10.3]$ at 95\% confidence)

  \begin{center}
    \raisebox{3.5mm}{
    \feynmandiagram [small,horizontal=a to b] {
    i1 [particle=\($g$\)]
        -- [gluon] t1 [dot]
        --  [anti fermion] a [dot]
        --  [anti fermion] t2 [dot]
        -- [anti fermion, edge label=\($t/b$\) ] t1,
    t2 -- [gluon] i2 [particle=\($g$\)],
    i1 -- [opacity=0] i2,
    a [label=\(\kappa_t\)] -- [scalar, edge label'=\($H$\)] b [dot, label=\(\kappa_\lambda\)],
    f1 [particle=\($H$\)] -- [scalar] b -- [scalar] f2 [particle=\($H$\)] -- [opacity=0] f1,
    };}
\hspace{1cm}
    \feynmandiagram [small, layered layout, horizontal=a to b] {
    % Draw the top and bottom lines
    i1 [particle=\($g$\)]
    -- [gluon] a [dot]
    -- [fermion] b [dot, label=\(\kappa_t\)]
    -- [scalar] f1 [particle=\($H$\)],
    i2 [particle=\($g$\)]
   -- [gluon] c [dot]
    -- [anti fermion] d [dot, label=below:\(\kappa_t\)]
    -- [scalar] f2 [particle=\($H$\)],
    % Draw the two internal fermion lines
    { [same layer] a -- [anti fermion, edge label=\($t/b$\)] c },
    { [same layer] b -- [fermion] d},
    };
\end{center}

  To find $\kappa_\lambda$ we need $HH$ events, and we can find them using jets!
\end{frame}


%%%%%%%%%%%%%%%%%%%%% SLIDE 3
{\setbeamertemplate{frame footer}{Image adapted from \href{https://en.wikipedia.org/wiki/B-tagging}{\alert{here}}}
\begin{frame}{Jets and Pairing}
  \begin{columns}[onlytextwidth]
    \begin{column}{0.76\textwidth}
      \centering
      \includegraphics[width=0.35\linewidth]{images/branching_ratios.png}\\
      $H \to 60\%\space b\bar{b} \to 2 \times b \text{ hadrons } \to 2 \times b\text{-jets}$
      \begin{itemize}
        \item \alert{Jets} are collections of particles with appx. the same direction
        \item ATLAS can't directly detect $H$ or $b$. Instead, use \alert{$b$-jets}, which can be directly detected (using secondary vertices)
        \item $b$-jet detection is not a perfect process (hence 437A report), and neither is \alert{pairing} -- choosing which jets came from which $H$
      \end{itemize}
    \end{column}
    \begin{column}{0.25\textwidth}
      \includegraphics[width=\linewidth]{images/jets_and_pairing.png}
    \end{column}
​  \end{columns}
\end{frame}
}

%%%%%%%%%%%%%%%%%%%%% SLIDE 4
\begin{frame}{The Background Modelling Problem, and the 2$b$RW solution}
  
  
  \begin{alertblock}{Potential problems}
		This is a potato
	\end{alertblock}
\end{frame}

%%%%%%%%%%%%%%%%%%%%% SLIDE 5
\begin{frame}{All caps}
	This frame uses the \texttt{allcaps} titleformat.

	\begin{alertblock}{Potential Problems}
		This titleformat is not as problematic as the \texttt{allsmallcaps} format, but basically suffers from the same deficiencies. So please have a look at the documentation if you want to use it.
	\end{alertblock}
\end{frame}

\section{Elements}

\begin{frame}[fragile]{Typography}
      \begin{verbatim}The theme provides sensible defaults to
\emph{emphasize} text, \alert{accent} parts
or show \textbf{bold} results.\end{verbatim}

  \begin{center}becomes\end{center}

  The theme provides sensible defaults to \emph{emphasize} text,
  \alert{accent} parts or show \textbf{bold} results.
\end{frame}

\begin{frame}{Font feature test}
  \begin{itemize}
    \item Regular
    \item \textit{Italic}
    \item \textsc{SmallCaps}
    \item \textbf{Bold}
    \item \textbf{\textit{Bold Italic}}
    \item \textbf{\textsc{Bold SmallCaps}}
    \item \texttt{Monospace}
    \item \texttt{\textit{Monospace Italic}}
    \item \texttt{\textbf{Monospace Bold}}
    \item \texttt{\textbf{\textit{Monospace Bold Italic}}}
  \end{itemize}
\end{frame}

\begin{frame}{Lists}
  \begin{columns}[T,onlytextwidth]
    \column{0.33\textwidth}
      Items
      \begin{itemize}
        \item Milk \item Eggs \item Potatos
      \end{itemize}

    \column{0.33\textwidth}
      Enumerations
      \begin{enumerate}
        \item First, \item Second and \item Last.
      \end{enumerate}

    \column{0.33\textwidth}
      Descriptions
      \begin{description}
        \item[PowerPoint] Meeh. \item[Beamer] Yeeeha.
      \end{description}
  \end{columns}
\end{frame}
\begin{frame}{Animation}
  \begin{itemize}[<+- | alert@+>]
    \item \alert<4>{This is\only<4>{ really} important}
    \item Now this
    \item And now this
  \end{itemize}
\end{frame}
\begin{frame}{Figures}
  \begin{figure}
    \newcounter{density}
    \setcounter{density}{20}
    \begin{tikzpicture}
      \def\couleur{alerted text.fg}
      \path[coordinate] (0,0)  coordinate(A)
                  ++( 90:5cm) coordinate(B)
                  ++(0:5cm) coordinate(C)
                  ++(-90:5cm) coordinate(D);
      \draw[fill=\couleur!\thedensity] (A) -- (B) -- (C) --(D) -- cycle;
      \foreach \x in {1,...,40}{%
          \pgfmathsetcounter{density}{\thedensity+20}
          \setcounter{density}{\thedensity}
          \path[coordinate] coordinate(X) at (A){};
          \path[coordinate] (A) -- (B) coordinate[pos=.10](A)
                              -- (C) coordinate[pos=.10](B)
                              -- (D) coordinate[pos=.10](C)
                              -- (X) coordinate[pos=.10](D);
          \draw[fill=\couleur!\thedensity] (A)--(B)--(C)-- (D) -- cycle;
      }
    \end{tikzpicture}
    \caption{Rotated square from
    \href{http://www.texample.net/tikz/examples/rotated-polygons/}{texample.net}.}
  \end{figure}
\end{frame}
\begin{frame}{Tables}
  \begin{table}
    \caption{Largest cities in the world (source: Wikipedia)}
    \begin{tabular}{lr}
      \toprule
      City & Population\\
      \midrule
      Mexico City & 20,116,842\\
      Shanghai & 19,210,000\\
      Peking & 15,796,450\\
      Istanbul & 14,160,467\\
      \bottomrule
    \end{tabular}
  \end{table}
\end{frame}
\begin{frame}{Blocks}
  Three different block environments are pre-defined and may be styled with an
  optional background color.

  \begin{columns}[T,onlytextwidth]
    \column{0.5\textwidth}
      \begin{block}{Default}
        Block content.
      \end{block}

      \begin{alertblock}{Alert}
        Block content.
      \end{alertblock}

      \begin{exampleblock}{Example}
        Block content.
      \end{exampleblock}

    \column{0.5\textwidth}

      \metroset{block=fill}

      \begin{block}{Default}
        Block content.
      \end{block}

      \begin{alertblock}{Alert}
        Block content.
      \end{alertblock}

      \begin{exampleblock}{Example}
        Block content.
      \end{exampleblock}

  \end{columns}
\end{frame}
\begin{frame}{Math}
  \begin{equation*}
    e = \lim_{n\to \infty} \left(1 + \frac{1}{n}\right)^n
  \end{equation*}
\end{frame}
\begin{frame}{Line plots}
  \begin{figure}
    \begin{tikzpicture}
      \begin{axis}[
        mlineplot,
        width=0.9\textwidth,
        height=6cm,
      ]

        \addplot {sin(deg(x))};
        \addplot+[samples=100] {sin(deg(2*x))};

      \end{axis}
    \end{tikzpicture}
  \end{figure}
\end{frame}
\begin{frame}{Bar charts}
  \begin{figure}
    \begin{tikzpicture}
      \begin{axis}[
        mbarplot,
        xlabel={Foo},
        ylabel={Bar},
        width=0.9\textwidth,
        height=6cm,
      ]

      \addplot plot coordinates {(1, 20) (2, 25) (3, 22.4) (4, 12.4)};
      \addplot plot coordinates {(1, 18) (2, 24) (3, 23.5) (4, 13.2)};
      \addplot plot coordinates {(1, 10) (2, 19) (3, 25) (4, 15.2)};

      \legend{lorem, ipsum, dolor}

      \end{axis}
    \end{tikzpicture}
  \end{figure}
\end{frame}
\begin{frame}{Quotes}
  \begin{quote}
    Veni, Vidi, Vici
  \end{quote}
\end{frame}

{%
\setbeamertemplate{frame footer}{My custom footer}
\begin{frame}[fragile]{Frame footer}
    \themename defines a custom beamer template to add a text to the footer. It can be set via
    \begin{verbatim}\setbeamertemplate{frame footer}{My custom footer}\end{verbatim}
\end{frame}
}

\begin{frame}{References}
  Some references to showcase [allowframebreaks] \cite{knuth92,ConcreteMath,Simpson,Er01,greenwade93}
\end{frame}

\section{Conclusion}

\begin{frame}{Summary}

  Get the source of this theme and the demo presentation from

  \begin{center}\url{github.com/matze/mtheme}\end{center}

  The theme \emph{itself} is licensed under a
  \href{http://creativecommons.org/licenses/by-sa/4.0/}{Creative Commons
  Attribution-ShareAlike 4.0 International License}.

  \begin{center}\ccbysa\end{center}

\end{frame}

{\setbeamercolor{palette primary}{fg=black, bg=yellow}
\begin{frame}[standout]
  Questions?
\end{frame}
}

\appendix

\begin{frame}[fragile]{Backup slides}
  Sometimes, it is useful to add slides at the end of your presentation to
  refer to during audience questions.

  The best way to do this is to include the \verb|appendixnumberbeamer|
  package in your preamble and call \verb|\appendix| before your backup slides.

  \themename will automatically turn off slide numbering and progress bars for
  slides in the appendix.
\end{frame}

\begin{frame}[allowframebreaks]{References}

  \bibliography{demo}
  \bibliographystyle{abbrv}

\end{frame}

\end{document}
